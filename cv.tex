%%%%%%%%%%%%%%%%%%%%%%%%%%%%%%%%%%%%%%%%%
% Medium Length Professional CV
% LaTeX Template
% Version 2.0 (8/5/13)
%
% This template has been downloaded from:
% http://www.LaTeXTemplates.com
%
% Original author:
% Trey Hunner (http://www.treyhunner.com/)
%
% Modified by:
% Guy John (http://rumblesan.com)
%
% Important note:
% This template requires the resume.cls file to be in the same directory as the
% .tex file. The resume.cls file provides the resume style used for structuring the
% document.
%
%%%%%%%%%%%%%%%%%%%%%%%%%%%%%%%%%%%%%%%%%

%----------------------------------------------------------------------------------------
%	PACKAGES AND OTHER DOCUMENT CONFIGURATIONS
%----------------------------------------------------------------------------------------

\documentclass{resume} % Use the custom resume.cls style

\ifdefined\iswebpage{}
  \usepackage[html]{tex4ht}
\fi
\usepackage[left=0.75in,top=0.6in,right=0.75in,bottom=0.6in]{geometry} % Document margins
\usepackage{hyperref}

\name{Guy John}
\address{07531 668965 \\ guy@peaklambda.com}

\begin{document}

%----------------------------------------------------------------------------------------
%	WORK EXPERIENCE SECTION
%----------------------------------------------------------------------------------------

\begin{rSection}{Experience}

  \begin{rExperience}{ustwo}{August 2015 to August 2017}{Fullstack Developer}{London}
  \item The agency nature of the company meant that the working environment was geared around medium length projects using a wide range of technologies. Because of the very product focussed mindset, pragmatic solutions that furthered clients objectives were generally preferred over `perfect' engineering.
  \item A large proportion of the work was centred around product discovery in combination with user testing. The aim being to validate or disprove ideas as quickly as possible.
  \item Teams worked in an agile manner and were always very cross functional, with developers, designers and product owners working together to prioritise and plan what work would be done next.
  \item The expectation was that teams would be entirely responsible for their infrastructure as well as their code. This meant that most of the projects involved setting up CI pipelines, managing deployments and integrating automated testing where appropriate.
  \item Projects were mainly React.js with either Node or Python backends. Some amount of Scala/Play framework used at points.
  \end{rExperience}

  %------------------------------------------------

  \begin{rExperience}{Pusher}{August 2014 to July 2015}{Platform Engineer}{London}
  \item Primarily involved in building a new distributed messaging bus for the Pusher software stack.
  \item Production codebase was heavily event driven, much of it utilising the Ruby event machine library.
  \item A very small development team meant that roles changed regularly, shifting between support, core engineering and ops.
  \item Very much a polyglot environment using mainly Haskell, Ruby and Javascript, but with many other languages used for client libraries and small tools.
  \end{rExperience}

  %------------------------------------------------

  \begin{rExperience}{Mind Candy}{April 2012 to July 2014}{Tools Software Engineer}{London}
  \item A highly varied role that involved building internal tools, maintaining third-party software, supporting product teams and making sure knowledge was shared.
  \item Projects were mostly web based tools to solve problems around deployments, dependency management, asset tracking, game community management and HR tasks.
  \item Also responsible for maintaining continuous integration systems and training other developers on how to use them.
  \item Team used the Kanban methodology because of the sometimes erratic nature of incoming work.
  \item Mix of Scala/Play Framework, Backbone/Angular and Python with PostgreSQL, Elastic Search and Cassandra databases.
  \end{rExperience}

  %------------------------------------------------

  \begin{rExperience}{Fidessa Plc}{Jan 2009 to April 2012}{Operational Engineer}{Woking, Surrey}
  \item Part of a small support and development team, mostly responsible for creating tools to aid long term capacity planning and server management.
  \item Began as a general operations engineer, but moved into a more development oriented role as tools became more widely used.
  \item Heavily involved with talking to tool users, gathering requirements and task prioritisation.
  \item Fairly standard Solaris, Apache, MySQL and PHP stack.
  \end{rExperience}

\end{rSection}

\pagebreak

%----------------------------------------------------------------------------------------
%	EDUCATION SECTION
%----------------------------------------------------------------------------------------

\begin{rSection}{Education}

  \begin{rUniversity}{University of Reading}{2005 to 2008}{BEng Electronic Engineering 2.1}
  \item Final year project was to create an analogue drum machine.
  \item Learnt to program in C, C++ and some amount of assembler, primarily targeting embedded micro-controllers,
  \item Also gained a solid grounding in DSP theory and engineering mathematics.
  \end{rUniversity}

  {\bf Eton College} \hfill {1999 to 2004} \\
  \begin{tabular}{@{} >{\bfseries}l @{\hspace{6ex}} l }
    A Levels & Physics, Maths and Further Maths \\
    AS Levels & I.T. and Electronics \\
    GCSEs & English, French, Latin, Chemistry, Physics, Biology, Maths, Electronics \\
  \end{tabular}

\end{rSection}


%----------------------------------------------------------------------------------------
%	TECHNICAL STRENGTHS SECTION
%----------------------------------------------------------------------------------------

\begin{rSection}{Technical Skills}

  \begin{tabular}{@{} >{\bfseries}l @{\hspace{6ex}} l }
    Languages & Haskell, Scala, Python, Javascript, Bash, C \\
    Frameworks & Play, Backbone, React, Express, Angular, Flask, Django \\
    Databases & MySQL, PostgreSQL, Redis \\
    Tools & Git, SVN, Docker, Puppet, General *nix sysadmin \\
    Infrastructure & AWS, Google Cloud, Kubernetes \\
    Management & Agile Development Practices, DevOps, Continuous Integration
  \end{tabular}

\end{rSection}

%----------------------------------------------------------------------------------------
%	INTERESTS SECTION
%----------------------------------------------------------------------------------------

\begin{rSection}{Interests}

  I'm heavily interested in the cross over of code with art and music, and have created or been involved with a number of projects along these lines. I am one half of LiveCodeLab, a duo doing live coded audio visual performances at venues ranging from boat and warehouse parties to the London Science Museum. At the end of 2013 I was involved with an installation in the Puerto Rico Museum of Art, writing the software for a computer controlled, reconstructed piano as part of the Lexus with the Arts program. I also produce music and play guitar, though primarily for my own pleasure.

  Outside of coding and art, I'm a keen rock climber, having now spent a significant amount of time climbing outside around the U.K. My decision to start contracting was mainly down to wanting more flexibility in how I work so I can spend more time falling off cliffs.

  Public speaking is something I enjoy and am always looking for opportunities to do more, whether that be in a technical capacity or otherwise. I've previously presented on subjects ranging from software tool development to language design for live coding environments.

\end{rSection}

\begin{rSection}{Links}

  A number of my projects, experiments and papers can be found on my website at \url{http://rumblesan.com/}

  My github account can be found at \url{http://github.com/rumblesan/}

\end{rSection}
%----------------------------------------------------------------------------------------

\end{document}
